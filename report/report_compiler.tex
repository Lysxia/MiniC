\documentclass[a4paper]{article}

\usepackage[french]{babel}
\usepackage[T1]{fontenc}
\usepackage[utf8]{inputenc}
\usepackage{fullpage}
\usepackage{pgf}
\usepackage{tikz}

\begin{document}

\title{Compilateur MiniC}
\author{Li-yao Xia}
\date{\today}

\maketitle
%\normalsize

\section{Commentaires généraux}

Comme je ne savais pas par où commencer,
j'ai choisi d'écrire le compilateur dans l'ordre des différentes étapes
de l'analyse. 

Lexeur -> Parseur -> Typeur

Les fichiers qui composent mon programme sont :
\begin{itemize}
  \item {\it lexer.mll}
  \item {\it ast.mli} - Décrit le type d'arbre de syntaxe abstraite décoré en
sortie du parseur.
  \item {\it parser.mly}
  \item {\it error.ml} - Fichier commun à tout le programme, de gestion
d'erreurs (affichage des messages d'erreur)
  \item {\it typing.ml/.mli}
  \item {\it main.ml}
\end{itemize}

\section{Choix et difficultés}

Je m'y suis pris assez tôt, donc j'avais globalement fini le
lexeur et le parseur avant qu'ils ne soient abordés dans le cours.
J'ai un peu revu le programme après chaque cours concerné pour effectuer
des simplifications et factoriser du code par des méthodes reprises du
cours. 

Un aspect de mon travail qui aurait pu me faire perdre beaucoup de temps est que 
j'ai écrit de grandes parties de mon programme sans prendre le temps de
vérifier leur correction. Des tests ont été faits seulement une fois le
parseur terminé (affichage de l'AST, fichier absent de l'archive), puis 
après que le typeur soit entièrement écrit.
Seules les erreurs de syntaxe étaient corrigées
pendant la majorité de la rédaction du projet. 

Finalement le débuggage n'a pas posé de problème majeur, probablement
parce que le projet (en tout cas cette partie) n'est pas complexe d'un
point de vue algorithmique ; il s'agissait surtout de retranscrire les
règles de typage dans un parcours d'arbre. Les fichiers de
tests semblent maintenant gérés correctement, même si quelques
imprécisions subsistent.

De plus, les conflits de lecture/réduction du parseur ont été en nombre
assez réduit dès la première compilation, et l'ajout d'une règle de
priorité n'en a laissé qu'un seul, ({\bf if} \dots {\bf else} \dots) 
ce qui apparaît plutôt miraculeux pour un premier usage de l'outil
menhir.

En cas d'ambiguïté dans le sujet, gcc a été la référence. (ainsi que pour
les messages d'erreurs que j'ai repris)

Des considérations de complexité et de simplicité m'ont amené à 
différentes tentatives d'optimisation, pour beaucoup futiles et prises de
tête :

\begin{itemize}
  \item convertir une boucle {\bf for} en boucle {\bf while} au moment du
parsing afin d'avoir un cas de moins à traiter dans la suite. Le code qui
résultait de cette "amélioration" n'était pas agréable du tout, j'ai
finalement abandonné cette idée.
  \item simplifier mon AST, d'où une certaine assymétrie à certains
niveaux par rapport à la grammaire fournie dans le sujet : je ne localise
pas les instructions du type {\bf expression;} ou les déclarations de
variables car ceci est déjà fait à un niveau plus bas dans l'arbre.
  \item à la sortie du typage, convertir les identifieurs en entiers
afin de diminuer les temps de recherche. Sur les machines modernes cette
conversion est plutôt accessoire mais j'ai tenu à le faire. 
\end{itemize}

En contraste avec ces maniaqueries, j'ai reporté le traitement de
certains cas particuliers à la fin de cette partie du projet.

\begin{itemize}
  \item caractères et chaînes de caractères : les séquences
d'échappement et les codes ascii n'étaient pas correctement lus par le
lexeur. On se contentait de recopier le code sans tenir compte des
caractères spéciaux. Le programme actuel supporte ces spécifications mais
le code me semble plutôt redondant dans sa forme actuelle.
  \item entiers : les {\bf int} en C sont représentés sur 32 bits tandis
qu'en OCaml ils le sont sur 31. La librairie {\bf Int32} m'a dispensé de
définir des types supplémentaires et d'écrire des fonctions de conversion 
\end{itemize}

Il reste enfin des points à améliorer :

\begin{itemize}
  \item {\it shift/reduce conflict} {\bf if} \dots {\bf else} \dots
\end{itemize}

\end{document}


